\documentclass[12pt,]{krantz}
\usepackage{lmodern}
\usepackage{amssymb,amsmath}
\usepackage{ifxetex,ifluatex}
\usepackage{fixltx2e} % provides \textsubscript
\ifnum 0\ifxetex 1\fi\ifluatex 1\fi=0 % if pdftex
  \usepackage[T1]{fontenc}
  \usepackage[utf8]{inputenc}
\else % if luatex or xelatex
  \ifxetex
    \usepackage{mathspec}
  \else
    \usepackage{fontspec}
  \fi
  \defaultfontfeatures{Ligatures=TeX,Scale=MatchLowercase}
    \setmonofont[Mapping=tex-ansi,Scale=0.7]{Source Code Pro}
\fi
% use upquote if available, for straight quotes in verbatim environments
\IfFileExists{upquote.sty}{\usepackage{upquote}}{}
% use microtype if available
\IfFileExists{microtype.sty}{%
\usepackage{microtype}
\UseMicrotypeSet[protrusion]{basicmath} % disable protrusion for tt fonts
}{}
\usepackage[unicode=true]{hyperref}
\PassOptionsToPackage{usenames,dvipsnames}{color} % color is loaded by hyperref
\hypersetup{
            pdftitle={Introduction to Data Science},
            pdfauthor={Hui Lin and Ming Li},
            colorlinks=true,
            linkcolor=Maroon,
            citecolor=Blue,
            urlcolor=Blue,
            breaklinks=true}
\urlstyle{same}  % don't use monospace font for urls
\usepackage{natbib}
\bibliographystyle{apalike}
\usepackage{color}
\usepackage{fancyvrb}
\newcommand{\VerbBar}{|}
\newcommand{\VERB}{\Verb[commandchars=\\\{\}]}
\DefineVerbatimEnvironment{Highlighting}{Verbatim}{commandchars=\\\{\}}
% Add ',fontsize=\small' for more characters per line
\usepackage{framed}
\definecolor{shadecolor}{RGB}{248,248,248}
\newenvironment{Shaded}{\begin{snugshade}}{\end{snugshade}}
\newcommand{\KeywordTok}[1]{\textcolor[rgb]{0.13,0.29,0.53}{\textbf{{#1}}}}
\newcommand{\DataTypeTok}[1]{\textcolor[rgb]{0.13,0.29,0.53}{{#1}}}
\newcommand{\DecValTok}[1]{\textcolor[rgb]{0.00,0.00,0.81}{{#1}}}
\newcommand{\BaseNTok}[1]{\textcolor[rgb]{0.00,0.00,0.81}{{#1}}}
\newcommand{\FloatTok}[1]{\textcolor[rgb]{0.00,0.00,0.81}{{#1}}}
\newcommand{\ConstantTok}[1]{\textcolor[rgb]{0.00,0.00,0.00}{{#1}}}
\newcommand{\CharTok}[1]{\textcolor[rgb]{0.31,0.60,0.02}{{#1}}}
\newcommand{\SpecialCharTok}[1]{\textcolor[rgb]{0.00,0.00,0.00}{{#1}}}
\newcommand{\StringTok}[1]{\textcolor[rgb]{0.31,0.60,0.02}{{#1}}}
\newcommand{\VerbatimStringTok}[1]{\textcolor[rgb]{0.31,0.60,0.02}{{#1}}}
\newcommand{\SpecialStringTok}[1]{\textcolor[rgb]{0.31,0.60,0.02}{{#1}}}
\newcommand{\ImportTok}[1]{{#1}}
\newcommand{\CommentTok}[1]{\textcolor[rgb]{0.56,0.35,0.01}{\textit{{#1}}}}
\newcommand{\DocumentationTok}[1]{\textcolor[rgb]{0.56,0.35,0.01}{\textbf{\textit{{#1}}}}}
\newcommand{\AnnotationTok}[1]{\textcolor[rgb]{0.56,0.35,0.01}{\textbf{\textit{{#1}}}}}
\newcommand{\CommentVarTok}[1]{\textcolor[rgb]{0.56,0.35,0.01}{\textbf{\textit{{#1}}}}}
\newcommand{\OtherTok}[1]{\textcolor[rgb]{0.56,0.35,0.01}{{#1}}}
\newcommand{\FunctionTok}[1]{\textcolor[rgb]{0.00,0.00,0.00}{{#1}}}
\newcommand{\VariableTok}[1]{\textcolor[rgb]{0.00,0.00,0.00}{{#1}}}
\newcommand{\ControlFlowTok}[1]{\textcolor[rgb]{0.13,0.29,0.53}{\textbf{{#1}}}}
\newcommand{\OperatorTok}[1]{\textcolor[rgb]{0.81,0.36,0.00}{\textbf{{#1}}}}
\newcommand{\BuiltInTok}[1]{{#1}}
\newcommand{\ExtensionTok}[1]{{#1}}
\newcommand{\PreprocessorTok}[1]{\textcolor[rgb]{0.56,0.35,0.01}{\textit{{#1}}}}
\newcommand{\AttributeTok}[1]{\textcolor[rgb]{0.77,0.63,0.00}{{#1}}}
\newcommand{\RegionMarkerTok}[1]{{#1}}
\newcommand{\InformationTok}[1]{\textcolor[rgb]{0.56,0.35,0.01}{\textbf{\textit{{#1}}}}}
\newcommand{\WarningTok}[1]{\textcolor[rgb]{0.56,0.35,0.01}{\textbf{\textit{{#1}}}}}
\newcommand{\AlertTok}[1]{\textcolor[rgb]{0.94,0.16,0.16}{{#1}}}
\newcommand{\ErrorTok}[1]{\textcolor[rgb]{0.64,0.00,0.00}{\textbf{{#1}}}}
\newcommand{\NormalTok}[1]{{#1}}
\usepackage{longtable,booktabs}
\IfFileExists{parskip.sty}{%
\usepackage{parskip}
}{% else
\setlength{\parindent}{0pt}
\setlength{\parskip}{6pt plus 2pt minus 1pt}
}
\setlength{\emergencystretch}{3em}  % prevent overfull lines
\providecommand{\tightlist}{%
  \setlength{\itemsep}{0pt}\setlength{\parskip}{0pt}}
\setcounter{secnumdepth}{5}
% Redefines (sub)paragraphs to behave more like sections
\ifx\paragraph\undefined\else
\let\oldparagraph\paragraph
\renewcommand{\paragraph}[1]{\oldparagraph{#1}\mbox{}}
\fi
\ifx\subparagraph\undefined\else
\let\oldsubparagraph\subparagraph
\renewcommand{\subparagraph}[1]{\oldsubparagraph{#1}\mbox{}}
\fi
\usepackage{booktabs}
\usepackage{longtable}
\usepackage[bf,singlelinecheck=off]{caption}

\setmainfont[UprightFeatures={SmallCapsFont=AlegreyaSC-Regular}]{Alegreya}

\usepackage{framed,color}
\definecolor{shadecolor}{RGB}{248,248,248}

\renewcommand{\textfraction}{0.05}
\renewcommand{\topfraction}{0.8}
\renewcommand{\bottomfraction}{0.8}
\renewcommand{\floatpagefraction}{0.75}

\renewenvironment{quote}{\begin{VF}}{\end{VF}}
\let\oldhref\href
\renewcommand{\href}[2]{#2\footnote{\url{#1}}}

\ifxetex
  \usepackage{letltxmacro}
  \setlength{\XeTeXLinkMargin}{1pt}
  \LetLtxMacro\SavedIncludeGraphics\includegraphics
  \def\includegraphics#1#{% #1 catches optional stuff (star/opt. arg.)
    \IncludeGraphicsAux{#1}%
  }%
  \newcommand*{\IncludeGraphicsAux}[2]{%
    \XeTeXLinkBox{%
      \SavedIncludeGraphics#1{#2}%
    }%
  }%
\fi

\makeatletter
\newenvironment{kframe}{%
\medskip{}
\setlength{\fboxsep}{.8em}
 \def\at@end@of@kframe{}%
 \ifinner\ifhmode%
  \def\at@end@of@kframe{\end{minipage}}%
  \begin{minipage}{\columnwidth}%
 \fi\fi%
 \def\FrameCommand##1{\hskip\@totalleftmargin \hskip-\fboxsep
 \colorbox{shadecolor}{##1}\hskip-\fboxsep
     % There is no \\@totalrightmargin, so:
     \hskip-\linewidth \hskip-\@totalleftmargin \hskip\columnwidth}%
 \MakeFramed {\advance\hsize-\width
   \@totalleftmargin\z@ \linewidth\hsize
   \@setminipage}}%
 {\par\unskip\endMakeFramed%
 \at@end@of@kframe}
\makeatother

\renewenvironment{Shaded}{\begin{kframe}}{\end{kframe}}

\newenvironment{rmdblock}[1]
  {
  \begin{itemize}
  \renewcommand{\labelitemi}{
    \raisebox{-.7\height}[0pt][0pt]{
      {\setkeys{Gin}{width=3em,keepaspectratio}\includegraphics{images/#1}}
    }
  }
  \setlength{\fboxsep}{1em}
  \begin{kframe}
  \item
  }
  {
  \end{kframe}
  \end{itemize}
  }
\newenvironment{rmdnote}
  {\begin{rmdblock}{note}}
  {\end{rmdblock}}
\newenvironment{rmdcaution}
  {\begin{rmdblock}{caution}}
  {\end{rmdblock}}
\newenvironment{rmdimportant}
  {\begin{rmdblock}{important}}
  {\end{rmdblock}}
\newenvironment{rmdtip}
  {\begin{rmdblock}{tip}}
  {\end{rmdblock}}
\newenvironment{rmdwarning}
  {\begin{rmdblock}{warning}}
  {\end{rmdblock}}

\usepackage{makeidx}
\makeindex

\urlstyle{tt}

\usepackage{amsthm}
\makeatletter
\def\thm@space@setup{%
  \thm@preskip=8pt plus 2pt minus 4pt
  \thm@postskip=\thm@preskip
}
\makeatother

\frontmatter

\title{Introduction to Data Science}
\author{\href{http://scientistcafe.com}{Hui Lin} and Ming Li}
\date{2017-11-27}

\usepackage{amsthm}
\newtheorem{theorem}{Theorem}[section]
\newtheorem{lemma}{Lemma}[section]
\theoremstyle{definition}
\newtheorem{definition}{Definition}[section]
\newtheorem{corollary}{Corollary}[section]
\newtheorem{proposition}{Proposition}[section]
\theoremstyle{definition}
\newtheorem{example}{Example}[section]
\theoremstyle{remark}
\newtheorem*{remark}{Remark}
\begin{document}
\maketitle

%\cleardoublepage\newpage\thispagestyle{empty}\null
%\cleardoublepage\newpage\thispagestyle{empty}\null
%\cleardoublepage\newpage
\thispagestyle{empty}
\begin{center}
%\includegraphics{images/dedication.pdf}
\end{center}

\setlength{\abovedisplayskip}{-5pt}
\setlength{\abovedisplayshortskip}{-5pt}

{
\hypersetup{linkcolor=black}
\setcounter{tocdepth}{3}
\tableofcontents
}
\listoftables
\listoffigures
\begin{Shaded}
\begin{Highlighting}[]
\CommentTok{# Sys.setenv(TZ="UTC")}
\KeywordTok{options}\NormalTok{(}\DataTypeTok{formatR.indent =} \DecValTok{2}\NormalTok{, }\DataTypeTok{width =} \DecValTok{55}\NormalTok{)}
\CommentTok{#bookdown::render_book("index.Rmd", "bookdown::gitbook")}
\CommentTok{#bookdown::render_book("index.Rmd", "bookdown::pdf_book")}
\end{Highlighting}
\end{Shaded}

\section*{Copyright Statement}\label{copyright-statement}


This work by Hui Lin and Ming Li is licensed under a
\href{https://creativecommons.org/licenses/by-nc-sa/3.0/us/}{Creative
Commons Attribution-NonCommercial-ShareAlike 3.0 United States License}.

Please note that this work is being written under a
\href{https://github.com/happyrabbit/IntroDataScience/blob/master/CONDUCT.md}{Contributor
Code of Conduct} and released under a
\href{https://creativecommons.org/licenses/by-nc-sa/3.0/us/}{CC-BY-NC-SA
license}. By participating in this project (for example, by submitting a
\href{https://github.com/happyrabbit/IntroDataScience/issues}{pull
request} with suggestions or edits) you agree to abide by its terms.

\section*{About the Authors}\label{about-the-authors}


\textbf{Hui Lin} is currently Data Scientist at DuPont Pioneer. She is a
leader within DuPont at applying advanced data science to enhance
Marketing and Sales effectiveness. She has been providing statistical
leadership for a broad range of predictive analytics and market research
analysis since 2013. She is the co-founder of Central Iowa R User Group,
blogger of scientistcafe.com and program Chair of Statistics in
Marketing Section of ASA for 2018. She enjoys making analytics
accessible to a broad audience and teaches tutorials and workshops for
practitioners on data science.

She holds MS and Ph.D.~in statistics from Iowa State University, BS in
mathematical statistics from Beijing Normal University.

\textbf{Ming Li} is currently a Sr. Data Scientist at Amazon. He was
Data Scientist at Wal-Mart and an Adjunct Faculty of Department and
Marketing and Business Analytics in TAMU -- Commerce. He is also the
Chair of Quality \& Productivity Section of ASA for 2016. He was a
Statistical Leader at General Electric Global Research Center and
Research Statistician at SAS Institute. He obtained his Ph.D.~in
Statistics from Iowa State University at 2010. With deep statistics
background and a few years' experience in data science, he has trained
and mentored numerous junior data scientist with different backgrounds
such as statistics, computer science, and business analytics.

\section*{Acknowledgements}\label{acknowledgements}


We want to give special thanks to Alex Shum and David Body for their
editing and comments on the sections of this book.

\section{The art of data science}\label{the-art-of-data-science}

Data science and data scientist have become buzz words. Allow me to
reiterate what you may have already heard a million times in the media:
\textbf{data scientists are in demand and demand continues to grow}. A
study by the McKinsey Global Institute concludes,

\begin{quote}
``a shortage of the analytical and managerial talent necessary to make
the most of Big Data is a significant and pressing challenge (for the
U.S.).''
\end{quote}

You may expect that statisticians and graduate students from traditional
statistics departments are great data scientist candidates. But the
situation is that the majority of current data scientists do not have a
statistical background. As David Donoho pointed out:

\begin{quote}
``statistics is being marginalized here; the implicit message is that
statistics is a part of what goes on in data science but not a very big
part.'' ( from
``\href{http://pages.cs.wisc.edu/~anhai/courses/784-fall15/50YearsDataScience.pdf}{50
years of Data Science}'').
\end{quote}

What is wrong? The activities that preoccupied statistics over centuries
are now in the limelight, but those activities are claimed to belong to
a new discipline and are practiced by professionals from various
backgrounds. Various professional statistics organizations are reacting
to this confusing situation. (Page 5-7, ``50 Years of Data Science'')
From those discussions, Donoho summarizes the main recurring ``Memes''
about data sciences:

\begin{enumerate}
\def\labelenumi{\arabic{enumi}.}
\tightlist
\item
  The `Big Data' Meme
\item
  The `Skills' Meme
\item
  The `Jobs' Meme
\end{enumerate}

The first two are linked together which leads to statisticians' current
position on data science. We assume everyone has heard the 3V (volume,
variety and velocity) definition of big data. The media hasn't taken a
minute break from touting ``big'' data. Data science trainees now need
the skills to cope with such big data sets. What are those skills? You
may hear about: Hadoop, system using Map/Reduce to process large data
sets distributed across a cluster of computers. The new skills are for
dealing with organizational artifacts of large-scale cluster computing
but not for better solving the real problem. A lot of data on its own is
worthless. It isn't the size of the data that's important. It's what you
do with it. The big data skills that so many are touting today are not
skills for better solving the real problem of inference from data.

Some media think they sense the trends in hiring and government funding.
We are transiting to universal connectivity with a deluge of data
filling telecom servers. But these facts don't immediately create a
science. The statisticians have been laying the groundwork of data
science for at least 50 years. Today's data science is an enlargement of
traditional academic statistics rather than a brand new discipline.

\subsection{What is data science?}\label{what-is-data-science}

This question is not new. When you tell people ``I am a data
scientist''. ``Ah, data scientist!'' Yes, who doesn't know that data
scientist is the sexist job in 21th century? If they ask further what is
data science and what exactly do data scientists do, it may effectively
kill the conversation.

Data Science doesn't come out of the blue. Its predecessor is data
analysis. Back in 1962, John Tukey wrote in ``The Future of Data
Analysis'':

\begin{quote}
For a long time I have thought I was a statistician, interested in
inferences from the particular to the general. But as I have watched
mathematical statistics evolve, I have had cause to wonder and to doubt.
\ldots{} All in all, I have come to feel that my central interest is in
data analysis, which I take to include, among other things: procedures
for analyzing data, techniques for interpreting the results of such
procedures, ways of planning the gathering of data to make its analysis
easier, more precise or more accurate, and all the machinery and results
of (mathematical) statistics which apply to analyzing data.
\end{quote}

It deeply shocked his academic readers. Aren't you supposed to present
something mathematically precise, such as definitions, theorems and
proofs? If we use one sentence to summarize what John said, it is:

\begin{quote}
data analysis is more than mathematics.
\end{quote}

\section{References}\label{references}

\bibliography{bibliography.bib}

\backmatter
\printindex

\end{document}
